\section*{Введение}

На сегодняшний день JavaScript является одним из самых
распространённых языков программирования~\cite{TIOBE,LANGPOP}.
Его популярность во многом объясняется тем, что это единственный
язык программирования, стандартизированный для выполнения в браузере.
При этом, за счёт растущей популярности веб-разработки а также
развития соответствующих технологий
(как различных вспомогательных библиотек,
так и сред исполнения) становится всё более популярным
использование веб-технологий (HTML, CSS, JavaScript)
как для разработки ``традиционных'' приложений с
графическим интерфейсом для ПК (WinJs, Node-Webkit),
мобильных устройств (PhoneGap, ReactNative) и серверных приложений
(Node.js).

Однако, как язык, JavaScript, по мнению многих разработчиков,
обладает рядом недостатков,
к примеру отсутствием статической типизации.
Поэтому в последнее время всё большую популярность
приобретают языки, которые могут компилироваться в JavaScript.
В качестве примера можно указать следующие проекты: TypeScript,
Dart, Scala.js, ClojureScript, и многие другие~\cite{TARGET_JS}\ldots

Kotlin --- статически типизированный язык, разрабатываемый
в компании JetBrains с 2010-го года. Поддерживается компиляция
в JVM-байткод и в JavaScript.

Подстановка функций (встраивание, inlining, inline expansion) --- распространённая
оптимизация в компиляторах, когда тело функции подставляется
в место вызова~\cite{WIKI_INLINE}. Применение такой оптимизации
имеет прямой эффект на производительности (не нужно выделять стрек
фрейм на вызов, копировать аргументы на стек), так и косвенный:
встроенное тело функции можно лучше оптимизировать.
К примеру, можно применить свёртку констант к аргументам функции,
после чего некоторые пути исполнения окажутся недостижимыми
ещё на этапе компиляции, и будут удалены.

Во многих языках встраивание вызова осуществляется по усмотрению
компилятора (эвристически). В языке Kotlin существует аннотация
\textbf{inline}. Вызов функции, помеченной такой аннотацией,
должен быть встроен, либо компилятор должен сообщить об ошибке.

В данной работе рассматривается реализация подстановки функций
при компиляции Kotlin в JavaScript, а также анализируется
влияние подстановки на производительность.
