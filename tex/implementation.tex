\section{Реализация}

Для простоты в некоторых примерах
структура генерируемого JavaScript-кода
упрощена для большей наглядности примеров.
Так, в настоящем коде функции всегда генерируются,
как поля объектов.

\subsection{Сохранение метаинформации}

При трансляции Kotlin AST в JavaScript AST теряется информация
об аннотациях, типах и т.д. Часть этих данных сохраняется
в вершины JavaScript AST при трансляции для использования
в процессе встраивания.

В частности в каждый элемент вызова сохраняется
флаг о том, нужно ли встраивать этот вызов.
Этот флаг истинен для вызовов функций, которые
помечены аннотацией \texttt{inline}, а также
для функциональных аргументов таких функций,
если они не помечены аннотацией \texttt{noinline}.

Также сохраняется информация о том, где
находилась вершина дерева в исходном коде для сообщений об ошибках.

\subsection{Разбиение выражений}

В грамматике JavaScript~\cite{EcmaScript} вызовы функций
могут использоваться и как выражения (expression), и
как утвержения (statement). Однако утверждения
(например цикл \texttt{for}) не могут быть частью выражений.

Если функция состоит только из утверждения \texttt{return},
то вызов функции может быть просто заменён на подвыражение
\texttt{return}. Однако в остальных случаях
тело функции может быть встроено только перед
утверждением, которое содержит этот вызов. После такого
встраивания может нарушиться поток исполнения программы,
так как подвыражения, которые должны выполнится
перед вызовом встраиваемой функции, будут вычислены после
выполнения тела этой функции.

Рассмотрим следующий пример:
\begin{listing}[H]
\begin{minted}[frame=lines]{javascript}
var x = foo() + bar();
\end{minted}
\caption{До встраивания bar.}
\end{listing}

\begin{listing}[H]
\begin{minted}[frame=lines]{javascript}
// тело bar
var x = foo() + bar_result;
\end{minted}
\caption{После встраивания bar.}
\end{listing}

Как видно на примере, после такого встраивания
\texttt{foo} выполнится после \texttt{bar}, хотя
в пользовательском коде было наоборот.

Поэтому перед встраиванием необходимо сохранить
все подвыражения, предшествующие вызову
встраиваемой функции, которые могут иметь побочные эффекты,
сохранить во временные переменные в том же порядке,
в каком они исполняются.

Не содержащими побочных эффектов могут считаться:
\begin{itemize}
  \item литералы (числа, строки, массивы);
  \item ссылки на локальные переменные (в общем случае ссылка на свойство
  может иметь побочный эффект);
  \item опрераторы, кроме присваивания, если
  их операнды не имеют побочных эффектов.
\end{itemize}

Все остальные вершины AST могут иметь побочные эффекты,
а значит должны быть извлечены.

Упрощённо алгоритм разбиения выражений можно описать так:
\begin{enumerate}
  \item Для каждой вершины AST запоминается:
    \begin{itemize}
      \item может ли вершина или её потомки иметь побочные эффекты;
      \item содержит ли вершина вызов встраиваемой функции.
    \end{itemize}
    Эту информацию можно получить за один обход дерева в глубину.
  \item Производится обход дерева в глубину, при этом в каждой
  вершине дети обходятся в порядке выполнения, определённым
  спецификацией EcmaScript~\cite{EcmaScript}. Если ребёнок
  содержит вызов встраиваемой функции, то перед его обходом
  все предыдущие дети, которые могут иметь побочный эффект,
  сохраняются во временные переменные.
  \item При разбиении необходимо учитывать ленивую семантику некоторых
  операторов:
  \begin{itemize}

  \item ``логическое или'' (\texttt{||})
  \begin{listing}[H]
  \begin{minted}[frame=lines]{javascript}
  var x = foo() || bar();
  \end{minted}
  \caption{До разбиения.}
  \end{listing}

  \begin{listing}[H]
  \begin{minted}[frame=lines]{javascript}
  var tmp = foo();
  if (!tmp) {
    tmp = bar();
  }
  var x = tmp;
  \end{minted}
  \caption{После разбиения.}
  \end{listing}

  \item ``логическое и'' (\texttt{\&\&})
  \begin{listing}[H]
  \begin{minted}[frame=lines]{javascript}
  var x = foo() && bar();
  \end{minted}
  \caption{До разбиения.}
  \end{listing}

  \begin{listing}[H]
  \begin{minted}[frame=lines]{javascript}
  var tmp = foo();
  if (tmp) {
    tmp = bar();
  }
  var x = tmp;
  \end{minted}
  \caption{После разбиения.}
  \end{listing}

  \item ``тернарный оператор'' (\texttt{?:})
  \begin{listing}[H]
  \begin{minted}[frame=lines]{javascript}
  var x = foo() ? bar() : foobar();
  \end{minted}
  \caption{До разбиения.}
  \end{listing}

  \begin{listing}[H]
  \begin{minted}[frame=lines]{javascript}
  var tmp;
  if (foo()) {
    tmp = bar();
  } else {
    tmp = foobar();
  }
  var x = tmp;
  \end{minted}
  \caption{После разбиения.}
  \end{listing}

  \end{itemize}
\end{enumerate}

\subsection{Процесс встраивания}

В качестве примера будем рассматривать встраивание
вызова функции \texttt{abs}:
\begin{listing}[H]
\begin{minted}[frame=lines]{javascript}
function abs(n) {
  if (n < 0) return -n;

  return n;
}

var x = abs(random(1000));
\end{minted}
\caption{Пример для иллюстрации процесса встраивания.}
\end{listing}

\begin{enumerate}
  \item Производится обход дерева в глубину, при этом перед
  встраиванием вызова производится обход вызываемой функции.
  Если во время обхода обнаружен цикл, то на каждый вызов цикла
  генерируется сообщение об ошибке.
  \item Все аргументы вызова, которые могут иметь побочный эффект,
  сохраняются во временные переменные.
  \begin{listing}[H]
  \begin{minted}[frame=lines]{javascript}
  var n = random(1000);
  var x = abs(n);
  \end{minted}
  \caption{Так как аргумент -- вызов функции \texttt{random} --
  может иметь побочный эффект, то для аргумента создаётся переменная n.}
  \end{listing}
  \item Параметры функции заменяются на аргументы. После этого
  вызовы параметров внутри функции можно встроить рекурсивно.
  \item \texttt{this} заменяется на соответствующий объект
  вызова.
  \item Все локальные имена изменяются таким образом, чтобы избежать
  пересечений с вызывающей функцией (к суффиксу добавляется число-счётчик).
  \item Тело функции помещается в блок с меткой \texttt{fun\_break}.
  \item Создаётся переменная для результата \texttt{fun\_result}.
  \item Вершины вида \texttt{return x;} заменяются на \texttt{fun\_result = x; break fun\_break;}.
  \item Получившееся тело функции встраивается перед последним посещённым
  утверждением.
  \item Вершина вызова заменяется на ссылку на переменную \texttt{fun\_result}.
  \begin{listing}[H]
  \begin{minted}[frame=lines]{javascript}
  var n = random(1000);
  var abs_result;
  abs_break: {
    if (n < 0) {
      abs_result = -n;
      break abs_break;
    }

    abs_result = n;
    break abs_break;
  }
  var x = abs_result;
  \end{minted}
  \caption{После встраивания.}
  \end{listing}
\end{enumerate}

В процессе для получения более читабельного и более компактного кода применяются
следующие эвристики:
\begin{itemize}
  \item Если тело функции состоит из одного \texttt{return},
  то вызов можно заменить на подвыражение \texttt{return} после шага 5.
  \item Если вызов функции содержится в утверждении вида \texttt{return bar()}
  (то есть вызов -- единственный ребёнок \texttt{return}), то
  тело функции можно встроить после шага 5.
  \item Если вызов функции содержится в утверждении вида \texttt{var x = bar()},
  то можно использовать \texttt{x} в качестве переменной для результата.
  \item Если функция состоит из некскольких утверждений, не содержащих
  \texttt{return}, за которыми следует \texttt{return}, то для такой функции
  не создаётся метка и переменная под результат, а
  вызов заменяется на подвыражение \texttt{return}.
\end{itemize}

Применение эвристик помогло уменьшить объём стандартной библиотеки
Kotlin после реализации подстановки на примерно на 4\%.

После встраивания удяляются подставленные анонимные функции
и локальные функции, если на них нет ссылок.

\subsection{Функции с замыканием}

При трансляции функции с замыканием внутри анонимной
функции на Kotlin в JavaScript
генерируется функция, которая принимает в качестве
параметров замыкаемые значения и возвращает анонимную функцию.
Таким образом в генерируемом JavaScript замыкания могут встречаться
лишь в функциях специального вида.

\begin{listing}[H]
\begin{minted}[frame=lines]{kotlin}
val a = 2
val b = arrayOf(1, 2, 3).map { x -> x * a }
\end{minted}
\caption{Анонимная функция с замыканием на Kotlin.}
\end{listing}

\begin{listing}[H]
\begin{minted}[frame=lines]{javascript}
function map_f(a) {
  return function(x) {
    return x * a;
  }
}

var a = 2;
var b = map([1, 2, 3], map_f(a));
\end{minted}
\caption{JavaScript код до встраивания.}
\end{listing}

Встраивание таких функций происходит в два этапа:
сначала происходит подстановка для внешней функции,
потом встраивается внутренняя функция с подставленными
``захваченными'' значениями.

Локальные функции в Kotlin также могут замыкать значения.
При этом, когда одна локальная функция вызывает другую,
то в генерируемом коде она начинает замыкать
соответствующую ссылку.

Если встраиваемую локальную функцию \texttt{inner\_a}
замыкает другая локальная функция \texttt{inner\_b},
то при встраивании \texttt{inner\_b} должна замыкать
переменные, замыкаемые \texttt{inner\_a}.

\begin{listing}[H]
\begin{minted}[frame=lines]{kotlin}
fun outer(): (Int)->Int {
  val x = 1

  [inline] fun inner_a(y: Int): Int = x + y

  fun inner_b(y: Int): Int = inner_a(y)

  return ::inner_b
}
\end{minted}
\caption{Локальные функции на Kotlin.}
\end{listing}

\begin{listing}[H]
\begin{minted}[frame=lines]{javascript}
function inner_a(x) {
  return function(y) {
    return x + y;
  }
}

function inner_b(f) {
  return function(y) {
    return f(y);
  }
}

function outer() {
  var x = 1;
  return inner_b(inner_a(x));
}
\end{minted}
\caption{Локальные функции после трансляции в JavaScript до встраивания:
\texttt{inner\_a} замыкает \texttt{x}, а \texttt{inner\_b} замыкает \texttt{inner\_a}.}
\end{listing}

\begin{listing}[H]
\begin{minted}[frame=lines]{javascript}
function inner_b(x, y) {
  return function(y) {
    return x + y;
  }
}

function outer() {
  var x = 1;
  return inner_b(x);
}
\end{minted}
\caption{Локальные функции после трансляции в JavaScript после встраивания:
\texttt{inner\_b} замыкает \texttt{x}.}
\end{listing}

\subsection{Встраивание функций из библиотек}

Для публично доступных встраиваемых функций для быстрого поиска
в файле генерируется тег в виде вызова функции
\texttt{defineInlineFunction}, первым аргументом которой
является полное имя функции, а вторым аргументом
тело самой функции (\texttt{defineInlineFunction}
просто возвращает второй аргумент).

\begin{listing}[H]
\begin{minted}[frame=lines]{javascript}
var utils = {
  sum: defineInlineFunction('utils.sum', function(x, y) {
    return x + y;
  })
};
\end{minted}
\caption{Пример скомпилированной библиотеки, которая содержит
встраиваемую функцию \texttt{sum}.}
\end{listing}

При встраивании вызова такой функции по имени в файле
ищется тег. Со следующего символа начинается
синтаксический анализ функции.

В качестве парсера используется Mozilla Rhino,
а также конвертер Rhino AST в JavaScript AST,
взятые из проекта GWT.

После получения JavaScript AST в теле
функции восстанавливается
метаинформация для функциональных аргументов. Далее встраивание
происходит, как и в общем случае.
